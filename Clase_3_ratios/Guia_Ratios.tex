% Options for packages loaded elsewhere
\PassOptionsToPackage{unicode}{hyperref}
\PassOptionsToPackage{hyphens}{url}
\PassOptionsToPackage{dvipsnames,svgnames,x11names}{xcolor}
%
\documentclass[
  11pt,
  letterpaper,
  DIV=11,
  numbers=noendperiod]{scrartcl}

\usepackage{amsmath,amssymb}
\usepackage{setspace}
\usepackage{iftex}
\ifPDFTeX
  \usepackage[T1]{fontenc}
  \usepackage[utf8]{inputenc}
  \usepackage{textcomp} % provide euro and other symbols
\else % if luatex or xetex
  \usepackage{unicode-math}
  \defaultfontfeatures{Scale=MatchLowercase}
  \defaultfontfeatures[\rmfamily]{Ligatures=TeX,Scale=1}
\fi
\usepackage{lmodern}
\ifPDFTeX\else  
    % xetex/luatex font selection
\fi
% Use upquote if available, for straight quotes in verbatim environments
\IfFileExists{upquote.sty}{\usepackage{upquote}}{}
\IfFileExists{microtype.sty}{% use microtype if available
  \usepackage[]{microtype}
  \UseMicrotypeSet[protrusion]{basicmath} % disable protrusion for tt fonts
}{}
\makeatletter
\@ifundefined{KOMAClassName}{% if non-KOMA class
  \IfFileExists{parskip.sty}{%
    \usepackage{parskip}
  }{% else
    \setlength{\parindent}{0pt}
    \setlength{\parskip}{6pt plus 2pt minus 1pt}}
}{% if KOMA class
  \KOMAoptions{parskip=half}}
\makeatother
\usepackage{xcolor}
\usepackage[margin=2.5cm]{geometry}
\setlength{\emergencystretch}{3em} % prevent overfull lines
\setcounter{secnumdepth}{5}
% Make \paragraph and \subparagraph free-standing
\makeatletter
\ifx\paragraph\undefined\else
  \let\oldparagraph\paragraph
  \renewcommand{\paragraph}{
    \@ifstar
      \xxxParagraphStar
      \xxxParagraphNoStar
  }
  \newcommand{\xxxParagraphStar}[1]{\oldparagraph*{#1}\mbox{}}
  \newcommand{\xxxParagraphNoStar}[1]{\oldparagraph{#1}\mbox{}}
\fi
\ifx\subparagraph\undefined\else
  \let\oldsubparagraph\subparagraph
  \renewcommand{\subparagraph}{
    \@ifstar
      \xxxSubParagraphStar
      \xxxSubParagraphNoStar
  }
  \newcommand{\xxxSubParagraphStar}[1]{\oldsubparagraph*{#1}\mbox{}}
  \newcommand{\xxxSubParagraphNoStar}[1]{\oldsubparagraph{#1}\mbox{}}
\fi
\makeatother


\providecommand{\tightlist}{%
  \setlength{\itemsep}{0pt}\setlength{\parskip}{0pt}}\usepackage{longtable,booktabs,array}
\usepackage{calc} % for calculating minipage widths
% Correct order of tables after \paragraph or \subparagraph
\usepackage{etoolbox}
\makeatletter
\patchcmd\longtable{\par}{\if@noskipsec\mbox{}\fi\par}{}{}
\makeatother
% Allow footnotes in longtable head/foot
\IfFileExists{footnotehyper.sty}{\usepackage{footnotehyper}}{\usepackage{footnote}}
\makesavenoteenv{longtable}
\usepackage{graphicx}
\makeatletter
\def\maxwidth{\ifdim\Gin@nat@width>\linewidth\linewidth\else\Gin@nat@width\fi}
\def\maxheight{\ifdim\Gin@nat@height>\textheight\textheight\else\Gin@nat@height\fi}
\makeatother
% Scale images if necessary, so that they will not overflow the page
% margins by default, and it is still possible to overwrite the defaults
% using explicit options in \includegraphics[width, height, ...]{}
\setkeys{Gin}{width=\maxwidth,height=\maxheight,keepaspectratio}
% Set default figure placement to htbp
\makeatletter
\def\fps@figure{htbp}
\makeatother

\KOMAoption{captions}{tableheading}
\makeatletter
\@ifpackageloaded{caption}{}{\usepackage{caption}}
\AtBeginDocument{%
\ifdefined\contentsname
  \renewcommand*\contentsname{Tabla de contenidos}
\else
  \newcommand\contentsname{Tabla de contenidos}
\fi
\ifdefined\listfigurename
  \renewcommand*\listfigurename{Listado de Figuras}
\else
  \newcommand\listfigurename{Listado de Figuras}
\fi
\ifdefined\listtablename
  \renewcommand*\listtablename{Listado de Tablas}
\else
  \newcommand\listtablename{Listado de Tablas}
\fi
\ifdefined\figurename
  \renewcommand*\figurename{Figura}
\else
  \newcommand\figurename{Figura}
\fi
\ifdefined\tablename
  \renewcommand*\tablename{Tabla}
\else
  \newcommand\tablename{Tabla}
\fi
}
\@ifpackageloaded{float}{}{\usepackage{float}}
\floatstyle{ruled}
\@ifundefined{c@chapter}{\newfloat{codelisting}{h}{lop}}{\newfloat{codelisting}{h}{lop}[chapter]}
\floatname{codelisting}{Listado}
\newcommand*\listoflistings{\listof{codelisting}{Listado de Listados}}
\makeatother
\makeatletter
\makeatother
\makeatletter
\@ifpackageloaded{caption}{}{\usepackage{caption}}
\@ifpackageloaded{subcaption}{}{\usepackage{subcaption}}
\makeatother

\ifLuaTeX
\usepackage[bidi=basic]{babel}
\else
\usepackage[bidi=default]{babel}
\fi
\babelprovide[main,import]{spanish}
% get rid of language-specific shorthands (see #6817):
\let\LanguageShortHands\languageshorthands
\def\languageshorthands#1{}
\ifLuaTeX
  \usepackage{selnolig}  % disable illegal ligatures
\fi
\usepackage{bookmark}

\IfFileExists{xurl.sty}{\usepackage{xurl}}{} % add URL line breaks if available
\urlstyle{same} % disable monospaced font for URLs
\hypersetup{
  pdftitle={Guía de Ratios Financieros --- Enunciados},
  pdfauthor={Prof.~Carlos Correa Íñiguez},
  pdflang={es},
  colorlinks=true,
  linkcolor={blue},
  filecolor={Maroon},
  citecolor={Blue},
  urlcolor={Blue},
  pdfcreator={LaTeX via pandoc}}


\title{Guía de Ratios Financieros --- Enunciados}
\usepackage{etoolbox}
\makeatletter
\providecommand{\subtitle}[1]{% add subtitle to \maketitle
  \apptocmd{\@title}{\par {\large #1 \par}}{}{}
}
\makeatother
\subtitle{Caso integral con interpretación --- DUOC UC}
\author{Prof.~Carlos Correa Íñiguez}
\date{2025-09-02}

\begin{document}
\maketitle

\renewcommand*\contentsname{Tabla de contenidos}
{
\hypersetup{linkcolor=}
\setcounter{tocdepth}{3}
\tableofcontents
}

\setstretch{1.15}
\section{Indicaciones generales}\label{indicaciones-generales}

Esta guía se resuelve en papel. En cada requerimiento: \textbf{fórmula →
sustitución → resultado → interpretación} (la interpretación tiene mayor
peso). Usa \textbf{promedios} de saldos para comparar con flujos (por
ejemplo, cuentas por cobrar promedio). Redondeo sugerido: \textbf{2
decimales} en ratios y días; \textbf{1 decimal} en porcentajes.

Ponderación sugerida: 40\% cálculo correcto + \textbf{60\%
interpretación} (coherencia, diagnóstico y acciones).

\section{1. Ejercicio largo --- Caso integral (ER +
Balance)}\label{ejercicio-largo-caso-integral-er-balance}

\textbf{Empresa:} \emph{TecnoAndes S.A.} (cifras en \textbf{MM\$})\\
\textbf{Objetivo:} Calcular e \textbf{interpretar} ratios de
\textbf{liquidez, actividad/rotación, endeudamiento, rentabilidad y
eficiencia}, y elaborar un \textbf{diagnóstico} con acciones.

\subsection{1.1 Estado de Resultados}\label{estado-de-resultados}

\begin{longtable}[]{@{}lrr@{}}
\toprule\noalign{}
Concepto & 2025 & 2024 \\
\midrule\noalign{}
\endhead
\bottomrule\noalign{}
\endlastfoot
Ventas brutas & 26.500 & 22.500 \\
(-) Devoluciones y descuentos & 1.500 & 1.500 \\
\textbf{Ventas netas} & \textbf{25.000} & \textbf{21.000} \\
(-) Costo de ventas & 16.000 & 12.600 \\
\textbf{Utilidad bruta} & \textbf{9.000} & \textbf{8.400} \\
(-) Gastos de venta y adm & 5.400 & 4.900 \\
(-) Depreciación y amortización & 1.100 & 900 \\
(+/-) Otros operativos (neto) & −200 & −100 \\
\textbf{Resultado operativo (EBIT)} & \textbf{2.300} & \textbf{2.500} \\
(-) Gastos financieros & 720 & 550 \\
\textbf{Resultado antes de impuestos} & \textbf{1.580} &
\textbf{1.950} \\
(-) Impuesto a la renta (20\%) & 316 & 390 \\
\textbf{Utilidad neta} & \textbf{1.264} & \textbf{1.560} \\
\end{longtable}

\subsection{1.2 Balance General}\label{balance-general}

\textbf{Activos}

\begin{longtable}[]{@{}lrr@{}}
\toprule\noalign{}
Activos (MM\$) & 2025 & 2024 \\
\midrule\noalign{}
\endhead
\bottomrule\noalign{}
\endlastfoot
Efectivo y equivalentes & 900 & 1.100 \\
Cuentas por cobrar comerciales & 3.600 & 2.400 \\
Inventarios & 5.200 & 3.900 \\
Otros activos corrientes & 300 & 200 \\
\textbf{Activo corriente} & \textbf{10.000} & \textbf{7.600} \\
Propiedades, planta y equipo (neto) & 6.500 & 6.100 \\
Intangibles (neto) & 900 & 1.100 \\
Otros activos no corrientes & 300 & 300 \\
\textbf{Activo no corriente} & \textbf{7.700} & \textbf{7.500} \\
\textbf{Total Activos} & \textbf{17.700} & \textbf{15.100} \\
\end{longtable}

\textbf{Pasivos y Patrimonio}

\begin{longtable}[]{@{}lrr@{}}
\toprule\noalign{}
Pasivos y Patrimonio (MM\$) & 2025 & 2024 \\
\midrule\noalign{}
\endhead
\bottomrule\noalign{}
\endlastfoot
Cuentas por pagar a proveedores & 3.200 & 2.000 \\
Deuda financiera CP & 1.500 & 900 \\
Provisiones y acumulaciones CP & 600 & 450 \\
Impuestos por pagar & 200 & 150 \\
\textbf{Pasivo corriente} & \textbf{5.500} & \textbf{3.500} \\
Deuda financiera LP & 4.900 & 3.700 \\
Otros pasivos no corrientes & 300 & 400 \\
\textbf{Pasivo no corriente} & \textbf{5.200} & \textbf{4.100} \\
\textbf{Total Pasivos} & \textbf{10.700} & \textbf{7.600} \\
Capital y aportes & 5.500 & 5.500 \\
Utilidades retenidas & 1.500 & 2.000 \\
\textbf{Patrimonio} & \textbf{7.000} & \textbf{7.500} \\
\textbf{Total Pasivos + Patrimonio} & \textbf{17.700} &
\textbf{15.100} \\
\end{longtable}

\subsection{1.3 Notas y supuestos}\label{notas-y-supuestos}

Usa saldos \textbf{promedio} para cuentas de balance cuando compares con
flujos. En particular:

\[
\mathrm{CxC}_{\text{prom}}=\frac{\mathrm{CxC}_{2024}+\mathrm{CxC}_{2025}}{2},\quad
\] \[
\mathrm{Inv}_{\text{prom}}=\frac{\mathrm{Inv}_{2024}+\mathrm{Inv}_{2025}}{2},\quad
\] \[
\mathrm{CxP}_{\text{prom}}=\frac{\mathrm{CxP}_{2024}+\mathrm{CxP}_{2025}}{2}.
\]

Supuestos adicionales:

\begin{itemize}
\tightlist
\item
  Ventas a crédito: \textbf{80\%} de las ventas netas.\\
\item
  Compras (aprox.): \$ \mathrm{Compras} = \mathrm{CV} +
  \Delta \mathrm{Inv} \$.\\
\item
  Compras a crédito para calcular la rotación de cuentas por pagar.\\
\item
  Redondea a \textbf{2 decimales} y usa \textbf{días} donde corresponda.
\end{itemize}

\subsection{1.4 Requerimientos de
cálculo}\label{requerimientos-de-cuxe1lculo}

\begin{enumerate}
\def\labelenumi{\arabic{enumi}.}
\tightlist
\item
  \textbf{Liquidez (2025):} Razón Corriente, Prueba Ácida, Caja/PC,
  Capital de Trabajo.\\
\item
  \textbf{Actividad/Rotación (2025):} Rotación de CxC y \textbf{días};
  Rotación de Inventario y \textbf{días}; Rotación de CxP y
  \textbf{días}; \textbf{Ciclo de conversión de caja (CCC)}.\\
\item
  \textbf{Endeudamiento (2025):} Deuda/Patrimonio (D/E), Deuda/Activo
  (\%), \textbf{Deuda financiera/Patrimonio}.\\
\item
  \textbf{Rentabilidad (2025):} Margen Bruto, Margen Neto, \textbf{ROA}
  y \textbf{ROE} (denominadores promedio), \textbf{Cobertura de
  intereses} (EBIT/Intereses).\\
\item
  \textbf{Eficiencia:} Rotación de activos totales (Ventas / Activo
  promedio).
\end{enumerate}

\subsection{1.5 Preguntas de interpretación (responder en
papel)}\label{preguntas-de-interpretaciuxf3n-responder-en-papel}

\begin{enumerate}
\def\labelenumi{\alph{enumi})}
\tightlist
\item
  \textbf{Quick \textless{} 1:} ¿Cuándo puede ser razonable y cuándo
  riesgoso? Relaciona con \textbf{Caja/PC} e \textbf{inventarios}.\\
\item
  \textbf{CCC:} ¿Qué componente explica su valor? Propón \textbf{dos
  palancas} para bajarlo ≥ \textbf{10 días}.\\
\item
  \textbf{Inventarios −15\%:} recalcula \textbf{días de inventario} y
  \textbf{CCC}. ¿Riesgos operativos?\\
\item
  \textbf{Pago +10 días:} efecto en \textbf{CCC}, relación con
  proveedores y liquidez.\\
\item
  \textbf{Incobrables 2\% de ventas a crédito:} impacto en
  \textbf{margen neto}, \textbf{coverage} y \textbf{ROE} (explica el
  canal contable).\\
\item
  \textbf{Tasa +200 pb:} recalcula \textbf{intereses},
  \textbf{coverage}, \textbf{margen neto} y \textbf{ROE}. Concluye sobre
  \textbf{riesgo}.\\
\item
  \textbf{Tendencia 2024→2025:} mejoras y deterioros (márgenes,
  cobertura, apalancamiento, capital de trabajo). ¿La rentabilidad
  compensa el riesgo?
\end{enumerate}

\subsection{1.6 Diagnóstico ejecutivo (10--12
líneas)}\label{diagnuxf3stico-ejecutivo-1012-luxedneas}

Integra resultados y respuestas en un \textbf{diagnóstico}: \textbf{2
fortalezas}, \textbf{2 debilidades} y \textbf{3 acciones} priorizadas
(crédito, inventarios, proveedores, deuda, etc.).




\end{document}
